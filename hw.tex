\hypertarget{ux434ux43eux43cux430ux448ux43dux435ux435-ux437ux430ux434ux430ux43dux438ux435-1}{%
\section{Домашнее задание
1}\label{ux434ux43eux43cux430ux448ux43dux435ux435-ux437ux430ux434ux430ux43dux438ux435-1}}

\hypertarget{ux432ux44bux431ux43eux440-ux440ux430ux441ux43fux440ux435ux434ux435ux43bux435ux43dux438ux439}{%
\subsection{Выбор
распределений}\label{ux432ux44bux431ux43eux440-ux440ux430ux441ux43fux440ux435ux434ux435ux43bux435ux43dux438ux439}}

Выбранные распределения:

\begin{itemize}
\tightlist
\item
  Дискретное: \emph{гипергеометрическое}
\item
  Непрерывное: \emph{нормальное}
\end{itemize}

\hypertarget{ux43eux43fux438ux441ux430ux43dux438ux435-ux43eux441ux43dux43eux432ux43dux44bux445-ux445ux430ux440ux430ux43aux442ux435ux440ux438ux441ux442ux438ux43a-ux440ux430ux441ux43fux440ux435ux434ux435ux43bux435ux43dux438ux439}{%
\subsection{Описание основных характеристик
распределений}\label{ux43eux43fux438ux441ux430ux43dux438ux435-ux43eux441ux43dux43eux432ux43dux44bux445-ux445ux430ux440ux430ux43aux442ux435ux440ux438ux441ux442ux438ux43a-ux440ux430ux441ux43fux440ux435ux434ux435ux43bux435ux43dux438ux439}}

\hypertarget{ux433ux438ux43fux435ux440ux433ux435ux43eux43cux435ux442ux440ux438ux447ux435ux441ux43aux43eux435-ux440ux430ux441ux43fux440ux435ux434ux435ux43bux435ux43dux438ux435}{%
\subsubsection{Гипергеометрическое
распределение}\label{ux433ux438ux43fux435ux440ux433ux435ux43eux43cux435ux442ux440ux438ux447ux435ux441ux43aux43eux435-ux440ux430ux441ux43fux440ux435ux434ux435ux43bux435ux43dux438ux435}}

Гипергеометрическое распределение - дискретное распределение,
описывающее вероятность события, при котором ровно \(k\) из \(n\)
случайно выбранных элементов окажутся \emph{помеченными}, при этом
выборка осуществляется из множества мощности \(N\), в котором
присутствует \(m\) помеченных элементов. Считается, что каждый из
элементов может быть выбран с одинаковой вероятностью \(\frac{1}{N}\).
Запишем это формально: \[\begin{gathered}
    N \in \mathbb{N},\ m \in \overline{0, N},\ n \in \overline{0, N},\\
    k \in \overline{0, n}
\end{gathered}\] Тогда \(HG(D, N, n)\) описывает вероятность события,
при котором ровно \(k\) из \(n\) элементов выборки окажутся
\emph{помеченными}: \[\begin{gathered}
    \left\{\xi \sim HG(N, m, n) \right\}\\
    \large\Updownarrow\\
    \left\{\Prob{\xi=k} = \frac{\binom{m}{k}\binom{N-m}{n-k}}{\binom{N}{n}}\right\}
\end{gathered}\]

\hypertarget{ux43cux430ux442ux435ux43cux430ux442ux438ux447ux435ux441ux43aux43eux435-ux43eux436ux438ux434ux430ux43dux438ux435-ux433ux438ux43fux435ux440ux433ux435ux43eux43cux435ux442ux440ux438ux447ux435ux441ux43aux43eux433ux43e-ux440ux430ux441ux43fux440ux435ux434ux435ux43bux435ux43dux438ux44f}{%
\paragraph{Математическое ожидание гипергеометрического
распределения}\label{ux43cux430ux442ux435ux43cux430ux442ux438ux447ux435ux441ux43aux43eux435-ux43eux436ux438ux434ux430ux43dux438ux435-ux433ux438ux43fux435ux440ux433ux435ux43eux43cux435ux442ux440ux438ux447ux435ux441ux43aux43eux433ux43e-ux440ux430ux441ux43fux440ux435ux434ux435ux43bux435ux43dux438ux44f}}

По определению, математическое ожидание случайной величины \(\xi\) --
это ее \(1^\text{й}\) начальный момент. Для начала, найдем
\(k^\text{й}\) начальный момент для \(\xi\): \[\Expect{\xi^r}
= \sum_{k=0}^{n} k^r \cdot \Prob{\xi=k}
= \sum_{k=0}^{n} k^r\frac{\binom{m}{k}\binom{N-m}{n-k}}{\binom{N}{n}}\]
Можем считать, что сумма берется при \(k\) от \(1\) до \(n\),так как
слагаемое при \(k=0\) будет равно \(0\). Заметим, что \[\begin{aligned}
    k\binom{m}{k} &= k \frac{m!}{k!(m-k)!} =\\
                &= k \frac{m \cdot (m-1)!}{k \cdot (k-1)! \cdot (m-k)!} =\\
                &= m \frac{(m-1)!}{(k-1)! \cdot (m-1 - (k-1))!} =\\
                &= m \binom{m-1}{k-1}
\end{aligned}\] и, как следствие, \[\binom{N}{n}
= \frac{1}{n} \cdot n \cdot \binom{N}{n}
= \frac{1}{n} N \binom{N-1}{n-1}\] Подставим TODO и TODO в TODO:
\[\Expect{\xi^r} = \frac{n \cdot m}{N}
\sum_{k=1}^{r-1} \frac{\binom{m-1}{k-1}\binom{N-m}{n-k}}{\binom{N-1}{n-1}}\]
Положим \(j := k-1\) и изменим индекс суммирования с на
\(j = \overline{0, n-1}\). Заметим, что
\(n - k = n - (j+1) = (n-1) - j\) и \(N - m = (N-1) - (m-1)\):
\[\Expect{\xi^r} = \frac{n \cdot m}{N} \textcolor{red}{\sum_{j=0}^{n-1} (j+1)^{r-1}
\frac{\binom{m-1}{j}\binom{(N-1) - (m-1)}{(n-1) - j}}{\binom{N-1}{n-1}}}\]
Заметим, что выделенная красным цветом часть выражения может быть
записана, как \(\Expect{(\theta+1)^{r-1}}\), где
\(\theta \sim HG(N-1, m-1, n-1)\). Следовательно,
\[\Expect{\xi^r} = \frac{n \cdot m}{N} \Expect{(\theta+1)^{r-1}}\] Таким
образом, \[\boxed{
    \Expect{\xi} = \frac{n \cdot m}{N}
}\]

\hypertarget{ux434ux438ux441ux43fux435ux440ux441ux438ux44f-ux433ux438ux43fux435ux440ux433ux435ux43eux43cux435ux442ux440ux438ux447ux435ux441ux43aux43eux433ux43e-ux440ux430ux441ux43fux440ux435ux434ux435ux43bux435ux43dux438ux44f}{%
\paragraph{Дисперсия гипергеометрического
распределения}\label{ux434ux438ux441ux43fux435ux440ux441ux438ux44f-ux433ux438ux43fux435ux440ux433ux435ux43eux43cux435ux442ux440ux438ux447ux435ux441ux43aux43eux433ux43e-ux440ux430ux441ux43fux440ux435ux434ux435ux43bux435ux43dux438ux44f}}

По определению дисперсии, \[\begin{aligned}
    \Var{\xi} &= \Expect{\left(\xi - \Expect{\xi}\right)^2} =\\
              &= \Expect{\xi^2} - \left(\Expect{xi}\right)^2
\end{aligned}\]

Выведем \(2^\text{й}\) начальный момент из TODO: \[\Expect{\xi^2}
= \frac{n \cdot m}{N}\Expect{\theta+1}
= \frac{n \cdot m}{N}\left(\frac{(n-1)(m-1)}{N-1}+1\right)\] Подставим
TODO и TODO в TODO: \[\begin{aligned}
    \Var{\xi} &= \Expect{\xi^2} - \left(\Expect{\xi}\right)^2 =\\
              &= \frac{n \cdot m}{N}\left(\frac{(n-1)(m-1)}{N-1}+1\right)
                    - \left(\frac{n \cdot m}{N}\right)^2=\\
              &= \frac{n \cdot m}{N}\left(\frac{(n-1)(m-1)}{N-1} + 1
                   - \frac{n \cdot m}{N}\right)
\end{aligned}\] Таким образом, \[\boxed{
    \Expect{\xi} = \frac{n \cdot m}{N}\left(\frac{(n-1)(m-1)}{N-1} + 1 - \frac{n \cdot m}{N}\right)
}\]

\hypertarget{ux43fux440ux43eux438ux437ux432ux43eux434ux44fux449ux430ux44f-ux444ux443ux43dux43aux446ux438ux44f-ux433ux438ux43fux435ux440ux433ux435ux43eux43cux435ux442ux440ux438ux447ux435ux441ux43aux43eux433ux43e-ux440ux430ux441ux43fux440ux435ux434ux435ux43bux435ux43dux438ux44f}{%
\paragraph{Производящая функция гипергеометрического
распределения}\label{ux43fux440ux43eux438ux437ux432ux43eux434ux44fux449ux430ux44f-ux444ux443ux43dux43aux446ux438ux44f-ux433ux438ux43fux435ux440ux433ux435ux43eux43cux435ux442ux440ux438ux447ux435ux441ux43aux43eux433ux43e-ux440ux430ux441ux43fux440ux435ux434ux435ux43bux435ux43dux438ux44f}}

По определению производящей функции, \[M_\xi(t) = \Expect{e^{t\xi}}\] То
есть, \[\begin{aligned}
    M_\xi(t) &= \sum_{k=0}^{n} e^{tk}\Prob{\xi=k} =\\
             &= \sum_{k=0}^{n} e^{tk}\frac{\binom{m}{k}\binom{N-m}{n-k}}{\binom{N}{n}}
\end{aligned}\]

TODO

\hypertarget{ux445ux430ux440ux430ux43aux442ux435ux440ux438ux441ux442ux438ux447ux435ux441ux43aux430ux44f-ux444ux443ux43dux43aux446ux438ux44f-ux433ux438ux43fux435ux440ux433ux435ux43eux43cux435ux442ux440ux438ux447ux435ux441ux43aux43eux433ux43e-ux440ux430ux441ux43fux440ux435ux434ux435ux43bux435ux43dux438ux44f}{%
\paragraph{Характеристическая функция гипергеометрического
распределения}\label{ux445ux430ux440ux430ux43aux442ux435ux440ux438ux441ux442ux438ux447ux435ux441ux43aux430ux44f-ux444ux443ux43dux43aux446ux438ux44f-ux433ux438ux43fux435ux440ux433ux435ux43eux43cux435ux442ux440ux438ux447ux435ux441ux43aux43eux433ux43e-ux440ux430ux441ux43fux440ux435ux434ux435ux43bux435ux43dux438ux44f}}

TODO

\hypertarget{ux433ux438ux441ux442ux43eux433ux440ux430ux43cux43cux430-ux432ux435ux440ux43eux44fux442ux43dux43eux441ux442ux435ux439-ux433ux438ux43fux435ux440ux433ux435ux43eux43cux435ux442ux440ux438ux447ux435ux441ux43aux43eux433ux43e-ux440ux430ux441ux43fux440ux435ux434ux435ux43bux435ux43dux438ux44f}{%
\paragraph{Гистограмма вероятностей гипергеометрического
распределения}\label{ux433ux438ux441ux442ux43eux433ux440ux430ux43cux43cux430-ux432ux435ux440ux43eux44fux442ux43dux43eux441ux442ux435ux439-ux433ux438ux43fux435ux440ux433ux435ux43eux43cux435ux442ux440ux438ux447ux435ux441ux43aux43eux433ux43e-ux440ux430ux441ux43fux440ux435ux434ux435ux43bux435ux43dux438ux44f}}

Построим гистограмму вероятностей для \(k \in \overline{0, n}\):

TODO

\hypertarget{ux444ux443ux43dux43aux446ux438ux44f-ux440ux430ux441ux43fux440ux435ux434ux435ux43bux435ux43dux438ux44f-ux433ux438ux43fux435ux440ux433ux435ux43eux43cux435ux442ux440ux438ux447ux435ux441ux43aux43eux433ux43e-ux440ux430ux441ux43fux440ux435ux434ux435ux43bux435ux43dux438ux44f}{%
\paragraph{Функция распределения гипергеометрического
распределения}\label{ux444ux443ux43dux43aux446ux438ux44f-ux440ux430ux441ux43fux440ux435ux434ux435ux43bux435ux43dux438ux44f-ux433ux438ux43fux435ux440ux433ux435ux43eux43cux435ux442ux440ux438ux447ux435ux441ux43aux43eux433ux43e-ux440ux430ux441ux43fux440ux435ux434ux435ux43bux435ux43dux438ux44f}}

По определению, функция распределения \(F_\xi(k) = \Prob{\xi < k}\).
Событие \(\{\xi < k\} = \bigcup\limits_{i=0}^{k-1}\{\xi=i\}\). События
\(\{\xi=i\}\; \forall i \in \overline{0, k-1}\) являются попарно
несовместными. То есть \(\forall i,j \in \overline{0, k-1}: i \neq j\)
выполняется \(\{\xi=i\}\large\cap\{\xi=j\}=\emptyset\). Из этого
следует, что \[\Prob{\xi < k} = \sum_{i=0}^{k-1}\Prob{\xi = i}\]
Подставим TODO в это выражение и получим: \[F_\xi(k)
= \sum_{i=0}^{k-1}\Prob{\xi = i}
= \sum_{i=0}^{k-1}\frac{\binom{m}{i}\binom{N-m}{n-i}}{\binom{N}{n}}\]

Построим график этой функции, учитывая, что аргументом \(k\) должно быть
натуральное число, не превосходящее \(n\):

TODO

\hypertarget{ux43dux43eux440ux43cux430ux43bux44cux43dux43eux435-ux440ux430ux441ux43fux440ux435ux434ux435ux43bux435ux43dux438ux435}{%
\subsubsection{Нормальное
распределение}\label{ux43dux43eux440ux43cux430ux43bux44cux43dux43eux435-ux440ux430ux441ux43fux440ux435ux434ux435ux43bux435ux43dux438ux435}}

Нормальное распределение - непрерывное распределение, описывающее
поведение величины отклонения измеряемого значения \(x\) от истинного
значения \(\mu\) (которое является математическим ожиданием) и в рамках
некоторого разброса \(\sigma\) (среднеквадратичного отклонения). Запишем
это формально: \[\begin{gathered}
    \left\{ \eta \sim N(\mu, \sigma^2) \right\}\\
    \Updownarrow\\
    \left\{\begin{gathered}
        F_\eta(x) = \Prob{\eta < x} = \int_{-\infty}^{x} f_\eta(x)dx,\\
        \text{где} f_\eta(x) = \frac{1}{\sigma\sqrt{2\pi}}e^{-\frac{(x-\mu)^2}{2\sigma^2}}
    \end{gathered}\right\}
\end{gathered}\] \(f_\eta(x)\) называется плотностью вероятности.

\hypertarget{ux43cux430ux442ux435ux43cux430ux442ux438ux447ux435ux441ux43aux43eux435-ux43eux436ux438ux434ux430ux43dux438ux435-ux43dux43eux440ux43cux430ux43bux44cux43dux43eux433ux43e-ux440ux430ux441ux43fux440ux435ux434ux435ux43bux435ux43dux438ux44f}{%
\paragraph{Математическое ожидание нормального
распределения}\label{ux43cux430ux442ux435ux43cux430ux442ux438ux447ux435ux441ux43aux43eux435-ux43eux436ux438ux434ux430ux43dux438ux435-ux43dux43eux440ux43cux430ux43bux44cux43dux43eux433ux43e-ux440ux430ux441ux43fux440ux435ux434ux435ux43bux435ux43dux438ux44f}}

Найдем математическое ожидание \(\eta \sim N(\mu, \sigma^2)\):
\[\begin{aligned}
    \Expect{\eta} &= \int_{-\infty}^{+\infty} x \cdot f_\eta(x)dx =\\
                  &= \int_{-\infty}^{+\infty} xe^{-\frac{(x-\mu)^2}{2\sigma^2}}dx =\\
                  &= \frac{1}{\sigma\sqrt{2\pi}} \int_{-\infty}^{+\infty} xe^{-\frac{(x-\mu)^2}{2\sigma^2}}dx
\end{aligned}\] Сделаем замену \(t = \frac{x-\mu}{\sqrt{2}\sigma}\):
\[\begin{aligned}
    \Expect{\eta} &= \frac{1}{\sigma\sqrt{2\pi}} \int_{-\infty}^{+\infty}(\sigma\sqrt{2}t + \mu)
                    e^{-t^2} d\left(\frac{x-\mu}{\sqrt{2}\sigma}\right) =\\
                  &= \frac{\sigma\sqrt{2}}{\sqrt{\pi}}\int_{-\infty}^{+\infty}te^{-t^2}dt
                    + \frac{\mu}{\sqrt{\pi}}\int_{-\infty}^{+\infty}e^{-t^2}dt =\\
                  &= \frac{\sigma\sqrt{2}}{\sqrt{\pi}}\left(\int_{-\infty}^{0}te^{-t^2}dt
                    - \int_{-\infty}^{0}te^{-t^2}dt\right) + \frac{\mu}{\sqrt{\pi}}\int_{-\infty}^{+\infty}e^{-t^2}dt =\\
                  &= \frac{\mu}{\sqrt{\pi}}\int_{-\infty}^{+\infty}e^{-t^2}dt
\end{aligned}\] Заметим, что получившееся выражение содержит интеграл,
который может быть сведен к интегралу
\href{https://ru.wikipedia.org/wiki/Гауссов_интеграл}{Эйлера-Пуассона}:
\[\int_{-\infty}^{+\infty}e^{-t^2}dt = 2\int_{0}^{+\infty}e^{-t^2}dt = \sqrt{\pi}\]
Таким образом, \[\boxed{
    \Expect{\eta} = \mu
}\]

\hypertarget{ux434ux438ux441ux43fux435ux440ux441ux438ux44f-ux43dux43eux440ux43cux430ux43bux44cux43dux43eux433ux43e-ux440ux430ux441ux43fux440ux435ux434ux435ux43bux435ux43dux438ux44f}{%
\paragraph{Дисперсия нормального
распределения}\label{ux434ux438ux441ux43fux435ux440ux441ux438ux44f-ux43dux43eux440ux43cux430ux43bux44cux43dux43eux433ux43e-ux440ux430ux441ux43fux440ux435ux434ux435ux43bux435ux43dux438ux44f}}

Подставим TODO в определение дисперсии TODO: \[\begin{aligned}
    \Var{\eta} &= \Expect{(\eta - \mu)^2} =\\
               &= \int_{-\infty}^{+\infty} (x-\mu)^2 \cdot f_{\eta}(x)dx =\\
               &= \int_{-\infty}^{+\infty}(x-\mu)^2 \frac{1}{\sigma\sqrt{2\pi}}e^{-\frac{(x-\mu)^2}{2\sigma^2}}dx =\\
               &= \frac{1}{\sigma\sqrt{2\pi}}\int_{-\infty}^{+\infty}(x-\mu)^2 e^{-\frac{(x-\mu)^2}{2\sigma^2}}dx
\end{aligned}\] Сделаем ту же замену переменной
\(t = \frac{x-\mu}{\sqrt{2}\sigma}\), тогда \(x = t\sqrt{2}\sigma+\mu\)
и: \[\begin{aligned}
    \Var{\eta} &= \frac{1}{\sigma\sqrt{2\pi}}
                \int_{-\infty}^{+\infty}(\sqrt{2}\sigma)^2 t^2 e^{-t^2}d(t\sqrt{2}\sigma+\mu) =\\
               &= \frac{2\sigma^2}{\sqrt{\pi}}\int_{-\infty}^{+\infty}t^2 e^{-t^2}dt
\end{aligned}\] Проинтегрируем по частям: \[\begin{aligned}
    \Var{\eta} &= \frac{\sigma^2}{\sqrt{\pi}}\int_{-\infty}^{+\infty}t 2t e^{-t^2} dt =\\
               &= \frac{\sigma^2}{\sqrt{\pi}}\left(\left. -t e^{-t^2} \right|_{-\infty}^{+\infty}
                 + \int_{-\infty}^{+\infty}e^{-t^2}dt\right)
\end{aligned}\] Здесь снова появляется интеграл
\href{https://ru.wikipedia.org/wiki/Гауссов_интеграл}{Эйлера-Пуассона}
и, в итоге, получаем: \[\boxed{
    \Var{\eta} = \sigma^2
}\] То есть, \(\sigma\) является среднеквадратичным отклонением.
